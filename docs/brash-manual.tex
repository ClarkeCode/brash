\documentclass{article}
\usepackage[colorlinks = true, urlcolor = blue, linkcolor = blue]{hyperref}
\usepackage{multirow}
\usepackage{listings}
\usepackage{color}

% Thank you Dave! https://www.davidpace.de/defining-custom-language-templates-for-latex-listings/
\lstdefinelanguage{brash} {
	morekeywords={
		var,
		if,
		else,
		while,
		func,
		print,
		true,
		false
	},
	sensitive=true, % keywords are case sensitive
	morecomment=[l]{//},
	%morecomment=[s]{/*}{*/}, % s is for start and end delimiter
	morestring=[b]" % defines that strings are enclosed in double quotes
}

\definecolor{darkGreen}{RGB}{63,127,95}
\lstset{
	frame=tb,
	language=brash,
	aboveskip=5mm,
	belowskip=5mm,
	showstringspaces=false,
	columns=flexible,
	basicstyle={\small\ttfamily},
	numbers=none,
	numberstyle=\tiny\color{blue},
	keywordstyle=\color{red},
	commentstyle=\color{blue},
	stringstyle=\color{darkGreen},
	breaklines=true,
	frame=trbl,
	%frameround=tttt,
	framesep=4pt,
	breakatwhitespace=true,
	tabsize=4
}

\title{Brash Reference Manual}
\author{Robert Clarke}
\date{\today}

\begin{document}

\maketitle
\tableofcontents
\clearpage

\section{A Foreword}
\subsection{Motivations}
\subsection{What Brash is}
\subsection{What Brash isn't}

\section{Introduction to Brash Syntax}

\subsection{Basic Types}
There are three basic types in Brash: \textbf{Number}, \textbf{Boolean}, and \textbf{String}.
\begin{itemize}
	\item Number values are base-10 double precision floating point numbers (Supports up to 6 decimal places)
	\item Boolean values are either \textbf{true} or \textbf{false}
	\item String values are series of textual characters that are enclosed by double-quotation marks
\end{itemize}
\begin{minipage}{\linewidth}
\lstinputlisting[caption={Basic Types}]{snippets/basic-types.brash}
\end{minipage}


\subsection{Variables}
To assign a name to a value, use variables for easy reference in your program.
To declare a variable use the \textbf{var} keyword, followed by an identifier, an equals sign, and the value you wish to bind to the variable.
After a variable is declared, you may reference the variable's value without restating the \textbf{var} keyword.
\begin{minipage}{\linewidth}
\lstinputlisting[caption={Variables}]{snippets/variables.brash}
\end{minipage}

\subsection{Expressions and Operators}
An expression is a series of values and operators which evaluate to a single value.
A majority of the operators in Brash are 'infix' operators which appear between values in the form "A OPERATOR B".
However, there are some prefix operators which appear before the value they modify.
\begin{itemize}
	\item Special operators
		\begin{itemize}
			\item Assignment = The assignment operator binds the variable identifier of its left operand to the value of its right operand
			\item Grouping () The grouping operators signal that the enclosed expression should be evaluated first
		\end{itemize}

	\item Arithmetic operators
		\begin{itemize}
			\item Addition +
			\item Subtraction -
			\item Multiplication *
			\item Division /
			\item Modulo \%
			\item Negation - (prefix)
		\end{itemize}

	\item Comparative operators
		\begin{itemize}
			\item Lesser \textless
			\item Lesser-Equals \textless=
			\item Greater \textgreater
			\item Greater-Equals \textgreater=
		\end{itemize}

	\item Identity operators
		\begin{itemize}
			\item Equality ==
			\item Inequality !=
		\end{itemize}

	\item Logical operators
		\begin{itemize}
			\item Logical NOT ! (prefix)
			\item Logical AND \&\&
			\item Logical OR $||$
			\item Logical XOR \^{}\^{}
		\end{itemize}
\end{itemize}

\subsubsection{Operator Precedence}
The order in which operators are evaluated are determined by their precedence.
Higher precedence operators are evaluated before lower precedence operators.

\begin{table}[!ht]
\begin{center}
	\caption{Precedence - High to Low}
\begin{tabular}{|c|l|l|}
	\hline
	Operator & Operator Name & Associativity \\
	\hline
	$( )$              & Group               & None  \\
	!                  & Logical NOT         & Right \\
%	\~{}               & Bitwise Complement  & Right \\
	-                  & Arithmetic Negation & Right \\

	*                  & Multiplication      & Left \\
	/                  & Division            & Left \\
	\%                 & Modulo              & Left \\

	+                  & Addition            & Left \\
	-                  & Subtraction         & Left \\

	\&\&               & Logical AND         & Left \\
	$||$               & Logical OR          & Left \\
	\^{}\^{}           & Logical XOR         & Left \\

	\textless          & Lesser              & None \\
	\textgreater       & Greater             & None \\
	\textless=         & Lesser-equal        & None \\
	\textgreater=      & Greater-equal       & None \\

	==                 & Equality            & None \\
	!=                 & Inequality          & None \\

	=                  & Assignment          & Right \\
	\hline
\end{tabular}
\end{center}
\end{table}

\clearpage

\subsection{Control Flow}
While you can make useful programs with the structures discussed so far, the magic really begins when your programs can take different actions based on certain conditions.

The \textbf{if} keyword must be followed by an expression that evaluates to a boolean value.
When the conditional expression is true, the statements within the if-block are executed.
When the conditional expression is false, the if-block is skipped.
\lstinputlisting[caption={If Statement}]{snippets/simple-if.brash}
If you have prior experience in the C family of programming languages, you may prefer to enclose your conditions with brackets:
\lstinputlisting[caption={If Statement (C-style)}]{snippets/simple-if-cstyle.brash}

To execute instructions when an \textbf{if}'s conditional expression is false, follow the \textbf{if} block with an \textbf{else} block.
\lstinputlisting[caption={If-Else}]{snippets/if-else.brash}

\subsection{Loops}
The \textbf{while} keyword is followed by an expression that evaluates to a boolean value.
The loop body follows after the conditional expression.
Once all the statements within the loop body have been executed, the conditional expression is re-evaluated.
If the conditional expression is false, skip the loop body and resume execution at the next statement.
\lstinputlisting[caption={While Loop}]{snippets/while-loop.brash}

%\begin{lstlisting}
%var myName = "Something"
%print "Something" + "New"
%if (20 > 10) {
%	print "Greg"
%}
%\end{lstlisting}

%\lstinputlisting[caption={Something New}]{test.brash}







\clearpage
\section{Bytecode Details}
Brash code gets compiled to bytecode before interpretation. If stored as a standalone file it is called a Brash Object File, and \textbf{.bro} is the preferred file extension.

\subsection{Numbers}
When encoding numbers into bytecode, they are encoded as 8-byte double-precision floating point numbers. Byte order is Big-Endian a.k.a. \href{https://www.rfc-editor.org/rfc/rfc1700}{TCP/IP Network Order.}

\subsection{Strings} \label{ssec:stringencode}
When encoding strings into bytecode, they are encoded as null-terminated strings.

\subsection{Functions}
When encoding a function definition into bytecode, functions are encoded as follows:
\begin{itemize}
	\item One-byte Function Definition Opcode
	\item Byte-length of function definition and function body: one 16-bit unsigned integer
	\item Encoded String of the function's name (See \ref{ssec:stringencode})
	\item Arity: one 8-bit unsigned integer
	\item If function \href{https://en.wikipedia.org/wiki/Arity}{arity} is 0 the following sub-items are omitted, otherwise each sub-item repeats a number of times equal to arity
		\begin{itemize}
			\item Encoded Type of n-th parameter
			\item Encoded String of the n-th parameter's name (See \ref{ssec:stringencode})
		\end{itemize}
	\item R-Arity (Return Arity): one 8-bit unsigned integer
	\item If function r-arity is 0 the following sub-item is omitted, otherwise the sub-item repeats a number of times equal to r-arity
		\begin{itemize}
			\item Encoded Type of n-th returned item
		\end{itemize}
	\item Function body: continue compilation as normal
\end{itemize}

\end{document}
