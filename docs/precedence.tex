\documentclass{article}
\usepackage[colorlinks = true, urlcolor = blue, linkcolor = blue]{hyperref}
\usepackage{multirow}

\title{Brash Reference Manual}
\author{Robert Clarke}
\date{\today}

\begin{document}

\maketitle
\tableofcontents
\clearpage

\section{Operator Precedence}
The order in which operators are evaluated are determined by their precedence.
Higher precedence operators are evaluated before lower precedence operators.

\begin{table}[!ht]
\begin{center}
	\caption{Precedence - High to Low}
\begin{tabular}{|c|l|l|}
	\hline
	Operator & Operator Name & Associativity \\
	\hline
	$( )$              & Group               & None  \\
	!                  & Logical NOT         & Right \\
%	\~{}               & Bitwise Complement  & Right \\
	-                  & Arithmetic Negation & Right \\

	*                  & Multiplication      & Left \\
	/                  & Division            & Left \\
	\%                 & Modulo              & Left \\

	+                  & Addition            & Left \\
	-                  & Subtraction         & Left \\

	\&\&               & Logical AND         & Left \\
	$||$               & Logical OR          & Left \\
	\^{}\^{}           & Logical XOR         & Left \\

	\textless          & Lesser              & None \\
	\textgreater       & Greater             & None \\
	\textless=         & Lesser-equal        & None \\
	\textgreater=      & Greater-equal       & None \\

	==                 & Equality            & None \\
	!=                 & Inequality          & None \\

	=                  & Assignment          & Right \\
	\hline
\end{tabular}
\end{center}
\end{table}




\clearpage
\section{Bytecode Details}

\subsection{Numbers}
When encoding numbers into bytecode, they are encoded as 8-byte double-precision floating point numbers. Byte order is Big-Endian a.k.a. \href{https://www.rfc-editor.org/rfc/rfc1700}{TCP/IP Network Order.}

\subsection{Strings} \label{ssec:stringencode}
When encoding strings into bytecode, they are encoded as null-terminated strings.

\subsection{Functions}
When encoding a function definition into bytecode, functions are encoded as follows:
\begin{itemize}
	\item One-byte Function Definition Opcode
	\item Encoded String of the function's name (See \ref{ssec:stringencode})
	\item Arity: one 8-bit unsigned integer
	\item If function \href{https://en.wikipedia.org/wiki/Arity}{arity} is 0 the following sub-items are omitted, otherwise each sub-item repeats a number of times equal to arity
		\begin{itemize}
			\item Encoded Type of n-th parameter
			\item Encoded String of the n-th parameter's name (See \ref{ssec:stringencode})
		\end{itemize}
	\item R-Arity (Return Arity): one 8-bit unsigned integer
	\item If function r-arity is 0 the following sub-item is omitted, otherwise the sub-item repeats a number of times equal to r-arity
		\begin{itemize}
			\item Encoded Type of n-th returned item
		\end{itemize}
	\item Function body: continue compilation as normal
\end{itemize}

\end{document}
