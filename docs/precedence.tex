\documentclass{article}
\usepackage[colorlinks = true, urlcolor = blue]{hyperref}
\usepackage{multirow}

\begin{document}

\section{Operator Precedence}
The order in which operators are evaluated are determined by their precedence.
Higher precedence operators are evaluated before lower precedence operators.

\begin{table}[h!]
\begin{center}
	\caption{Precedence - High to Low}
\begin{tabular}{|c|l|l|}
	\hline
	Operator & Operator Name & Associativity \\
	\hline
	$( )$              & Group               & None  \\
	!                  & Logical NOT         & Right \\
	\~{}               & Bitwise Complement  & Right \\
	-                  & Arithmetic Negation & Right \\

	*                  & Multiplication      & Left \\
	/                  & Division            & Left \\
	\%                 & Modulo              & Left \\

	+                  & Addition            & Left \\
	-                  & Subtraction         & Left \\

	\&\&               & Logical AND         & Left \\
	$||$               & Logical OR          & Left \\
	\^{}\^{}           & Logical XOR         & Left \\

	\textless          & Lesser              & None \\
	\textgreater       & Greater             & None \\
	\textless=         & Less-equal          & None \\
	\textgreater=      & Greater-equal       & None \\

	==                 & Equality            & None \\
	!=                 & Inequality          & None \\

	=                  & Assignment          & Right \\
	\hline
\end{tabular}
\end{center}
\end{table}

\section{Bytecode Details}
When encoding numbers into bytecode, they are encoded as 8-byte double-precision floating point numbers. Byte order is Big-Endian aka TCP/IP Network Order.

\end{document}
